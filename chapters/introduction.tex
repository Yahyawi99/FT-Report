\chapter*{Introduction générale}
\addcontentsline{toc}{chapter}{Introduction générale}
Dans le cadre de ma formation de licence en Sciences Mathématique et Informatique spécialité Bases de Données, la réalisation d'un stage de fin d'études constitue une étape essentielle. Ce stage permet de confronter les connaissances théoriques acquises à l'université avec les exigences et les pratiques du monde professionnel. Il représente également une opportunité pour développer des compétences techniques, méthodologiques et relationnelles dans un environnement réel.\\[5mm]
J'ai effectué mon stage au sein de l'entreprise française \textbf{\textcolor{ftRed}{FeverTokens}}, spécialisée dans les solutions innovantes intégrant le Web 3.0, le Cloud et la Blockchain. Ce stage, d'une durée de 6 mois, avait pour objectif de participer à l'étude et au développement d'une application Web 3.0 avec un backend hybride Cloud/Blockchain. Cette expérience m'a permis d'élargir mes connaissances, de renforcer mes aptitudes en développement logiciel et de mieux comprendre les enjeux liés aux nouvelles technologies.
\\[5mm]
Ce stage s'est déroulé en mode \textbf{distanciel}, ce qui a constitué un défi particulier mais aussi une expérience enrichissante. Grâce à des réunions quotidiennes de type \textit{stand-up}, ainsi qu'à l'utilisation d'outils de communication et de collaboration sophistiqués tels que Git, GitHub et diverses plateformes de visioconférence, nous avons pu maintenir une excellente synchronisation. L'\' esprit de collaboration de l'équipe a permis de surmonter les contraintes liées à la distance et de garantir un suivi régulier et efficace des tâches.
\\[5mm]
Au début de mon stage,  j'ai été confronté à un défi majeur : mon manque de connaissances dans des domaines clés tels que la \textbf{Blockchain}, le \textbf{Web 3.0} et les \textbf{Smart Contracts}. Cette situation a représenté une source de motivation supplémentaire, m'incitant à apprendre rapidement et à m'adapter afin de pouvoir contribuer efficacement aux missions confiées.\\[5mm]
Ce stage m'a aussi permis de renforcer mes compétences en \textbf{développement full-stack} et en \textbf{développement logiciel en général}. J'ai appris à utiliser de nouveaux outils et technologies, tout en améliorant ma capacité à produire un code plus propre, structuré et maintenable. Cela s'est concrétisé notamment par l'adoption d'une \textbf{structure monorepo}, ce qui a facilité la gestion et le lien entre les différents packages du projet, tout en assurant une cohérence et une efficacité accrues.\\[5mm]
Dans ce rapport, je vais détailler les différentes étapes de mon stage, les missions qui 
m'ont été confiées, les défis rencontrés et les solutions apportées. Je présenterai également les technologies utilisées, les compétences acquises et les perspectives d'avenir dans le domaine du Web 3.0 et de la Blockchain.\\[5mm]