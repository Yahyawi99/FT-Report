\chapter*{Chapitre 3: Développement \& Analyse}
\addcontentsline{toc}{chapter}{Chapitre 3: Développement \& Analyse}
\stepcounter{chapter}

\section{Introduction}
Ce chapitre décrit en détail le travail réalisé au cours de mon stage, en mettant l'accent sur le projet principal \textbf{TEI-TDFD}, les tâches techniques additionnelles, ainsi que les présentations hebdomadaires organisées avec l'équipe. 
L'objectif est d'exposer à la fois le processus de développement et l'analyse des apprentissages tirés de chaque contribution.

\section{Projet Principal: TEI-TDFD}
\subsection{Objectif et Contexte}
% Présenter le but global du projet et sa place dans l'entreprise

\subsection{Mon Rôle et Responsabilités}
% Détailler vos missions spécifiques dans le projet

\subsection{Mise en œuvre et Processus}
% Décrire les étapes de développement, les choix technologiques, les difficultés rencontrées et les solutions apportées

\subsection{Résultats et Contributions}
% Résumer ce qui a été accompli et votre apport concret au projet

\section{Contributions Techniques Additionnelles}
\subsection{Tâche A : [Nom ou courte description]}
% Explication de la tâche, de son objectif, et de votre intervention

\subsection{Tâche B : [Nom ou courte description]}
% Idem pour la deuxième tâche

% Ajouter plus de sous-sections si nécessaire

\section{Suivi Hebdomadaire et Présentations}
\subsection{Format et Objectif des Présentations}
% Expliquer la logique : sujets donnés par les encadrants, but pédagogique, rôle dans la formation des stagiaires

\subsection{Sujets Traités et Retours Obtenus}
% Liste/résumé des thèmes présentés et feedback reçu

\subsection{Rôle dans la Communication d'Équipe}
% Expliquer comment ces présentations renforçaient la cohésion et le partage des connaissances

\section{Conclusion}
Ce chapitre a permis de détailler les contributions principales et secondaires réalisées durant le stage, ainsi que les activités de suivi et de présentation qui ont favorisé à la fois l'avancement des projets et mon développement professionnel.
